%
% Copyright 2014, NICTA
%
% This software may be distributed and modified according to the terms of
% the BSD 2-Clause license. Note that NO WARRANTY is provided.
% See "LICENSE_BSD2.txt" for details.
%
% @TAG(NICTA_BSD)
%

%%%%%%%%%%%%%%%%%%%%%%%%%%%%%%%%%%%%%%%%%%%%%%%%%%%%%%%%%%%%
% For technical reports, manuals etc:
%  - use ``report''
%%%%%%%%%%%%%%%%%%%%%%%%%%%%%%%%%%%%%%%%%%%%%%%%%%%%%%%%%%%%
\documentclass[a4paper,11pt,twoside]{report}
\usepackage[colour,nictaonly]{disy}

%%% $Id: paper.tex,v 1.22 2008-07-19 08:14:55 gernot Exp $

% Setting this to true turns on the ``draft'' watermark
\newif \ifDraft         \Drafttrue

%%% Hyperlinks embed links for references in the PDF output (both for
%%% internal as well as external links).
\newif \ifhyperlinks    \hyperlinkstrue
%%% When using this you should use \autoref{label} instead of
%%% Section~\ref{label}. This will make ``Section'' part of the link, not
%%% just the number.

% Easy control over page margins. You can use ``right'', ``top'', ...
\usepackage[margin=25mm]{geometry}
%\usepackage[margin=25mm,left=33mm]{geometry}	% better for double-sided reports

\usepackage{graphicx}
\setkeys{Gin}{keepaspectratio=true,clip=true,draft=false,width=\linewidth}
\graphicspath{{./imgs/}}
\usepackage{times,cite,url,fancyhdr}
\urlstyle{rm}
\renewcommand{\ttdefault}{cmtt}	% CM rather than courier for \tt

\ifDraft
  \usepackage{draftcopy}
  \newcommand{\Comment}[1]{\textbf{\textsl{#1}}}
  \newcommand{\FIXME}[1]{\textbf{\textsl{FIXME: #1}}}
  \date{\small$\relax$Id: paper.tex,v 1.22 2008-07-19 08:14:55 gernot Exp $\relax$}
\else
  \newcommand{\Comment}[1]{\relax}
  \newcommand{\FIXME}[1]{\relax}
  \date{}
\fi

\ifhyperlinks
  \usepackage[pdftex,pagebackref,hyperindex,bookmarks]{hyperref}
  \hypersetup{linktocpage} %,colorlinks}
\else   
  \providecommand{\href}[2]{\url{#2}}
\fi

\pagestyle{fancyplain}
\lhead[\fancyplain{}{\sl\thepage}]{\fancyplain{}{\sl\rightmark}}
\chead{}
\rhead[\fancyplain{}{\sl\leftmark}]{\fancyplain{}{\sl\thepage}}
\lfoot[\fancyplain{\sl\thepage}{}]{}
\cfoot{\ifDraft\textsf{NICTA Confidential}\fi}
\rfoot[]{\fancyplain{\sl\thepage}{}}

\newcommand{\skcapdl}{SK-CapDL}
\newcommand{\skentity}[1]{\emph{#1}}
\newcommand{\term}[1]{{\it #1}}
\newcommand{\code}[1]{\\[+4pt]
{{\tt \hphantom{    } #1}}\\[+4pt]
}
\newcommand{\todo}[1]{{\bf \emph{TODO: #1}}}
\usepackage{listings}
\lstset{language=C,keywordstyle=\bf,morekeywords={size_t,off_t},basicstyle=\ttfamily\small}

\begin{document}

  \ifhyperlinks
    %% Capitalisation of cross references
    \renewcommand{\chapterautorefname}{Chapter}
    \renewcommand{\sectionautorefname}{Section}
    \renewcommand{\subsectionautorefname}{Section}
    \renewcommand{\subsubsectionautorefname}{Section}
    \renewcommand{\appendixautorefname}{Appendix}
    \renewcommand{\Hfootnoteautorefname}{Footnote}
    %% Commands for index
    \newcommand{\Htextbf}[1]{\textbf{\hyperpage{#1}}}
  \fi

  \title{\skcapdl~Manual}

  \author{Matthew Fernandez, Gerwin Klein}
  \AuthorEmail{matthew.fernandez@nicta.com.au}
  \date{March 2013}

  \maketitle
  \urlstyle{sf}
  \thispagestyle{empty}

  \begin{abstract}
    \pagenumbering{roman}
This document describes the usage of a tool for generating CapDL
\cite{Kuz_KLW_10} and code stubs for building separation kernel systems on
seL4 \cite{Klein_EHACDEEKNSTW_09}.
It was developed as part of a collaboration with \href{http://corp.galois.com/}{Galois Inc}.

    \vfill

    \noindent
    \copyright~2013 NICTA
  \end{abstract}
  \setcounter{page}{1}


  \tableofcontents

  \cleardoublepage
  \setcounter{page}{1}
  \pagenumbering{arabic}

  \chapter{\label{s:build}Building \skcapdl}
The instructions in this section assume your environment is Ubuntu Linux.
Steps should work on any Linux or Mac OS machine with minimal changes.

  \section{\label{sec:dependencies}Dependencies}
The \skcapdl~tool is written in Haskell, so you will need a Haskell compiler
in order to build it.
We recommend the Glasgow Haskell Compiler \cite{GHC}.
To install this, run:
\code{
    sudo apt-get install ghc
}
Several extra libraries are required, which are available through Cabal, the
Haskell package manager.
To install Cabal, run:
\code{
    sudo apt-get install cabal
}
The libraries required are MissingH, elf and xml. To install these:
\code{
    cabal install MissingH

    cabal install elf

    cabal install xml
}

  \section{\label{sec:compilation}Compilation}
The source code for the tool itself is in the directory {\tt src}.
Compile it using the make command:
\code{
    cd src

    make
}

  \chapter{\label{s:usage}Usage}
The \skcapdl~tool is designed to run in three phases generating three distinct
outputs:
\begin{enumerate}
  \item Stub code generation, during which C source and header files are
    generated;
  \item Policy generation, during which a CapDL specification for initialising
    the system is generated; and
  \item Scheduler configuration generation, during which a domain schedule for
    the kernel is generated.
\end{enumerate}
Which of these phases is run is determined by the {\tt --output} command-line
argument.

  \section{\label{sec:input}Input Specification}
The input to the \skcapdl~tool is an XML specification describing the
separation configuration of the system.
This specification describes the scheduling constraints of the system, as well
as shared memory regions (referred to as \term{segments}) and their mappings to
accessible regions within the cells.

An example specification of a simple system is the following:
\vspace{6pt}
{\lstset{language=xml,frame=single,keywordstyle=\bf,basicstyle=\ttfamily\small,morekeywords={cell,layout,use,segment}}\lstinputlisting{sk-example.xml}}
\vspace{6pt}
This describes three segments, \skentity{seg1}, \skentity{seg2} and
\skentity{seg3}, of sizes 1024, 1024 and
4096 bytes respectively.
There are two cells in this system, \skentity{cell1} and \skentity{cell2}.
In the context of \skentity{cell1}, \skentity{seg1} is accessible read-only as a
symbol and functions parameterised by the alias \skentity{input}.
This segment is accessible in the context of \skentity{cell2} as a write-only
region parameterised by the alias \skentity{output}.
Conversely the segment \skentity{seg2} is accessible write-only to
\skentity{cell1} with the alias \skentity{output} and read-only to
\skentity{cell2} with the alias \skentity{input}.
The final segment \skentity{seg3} is accessible for reading and writing by
\skentity{cell2} with the alias \skentity{misc}.

A generalised specification has a single layout element, but can contain an
arbitrary number of segments and cells.
It is expected to conform to the following XML schema:
\vspace{6pt}
{\lstset{language=xml,frame=single,basicstyle=\ttfamily\small}\lstinputlisting{sk-schema.xsd}}

  \section{\label{sec:codegen}Code Generation}
Given an XML specification, {\tt spec.xml} the tool can be executed as follows
to generate source code stubs for a cell, \skentity{cell1}:
\code{
    sk-capdl --xml spec.xml --output source --cell cell1
}
This generates a source file named ``cell1\_driver.c''.
This file provide an entry point, {\tt main}, and functions for accessing
each segment available to \skentity{cell1}.
To generate a header for including in other C files change the option passed to
{\tt --output}:
\code{
    sk-capdl --xml spec.xml --output header --cell cell1
}
This generates a header file named ``cell1\_driver.h''.

For the example specification shown previously, the following functions are
generated for the region \skentity{input} in \skentity{cell1}:
\begin{lstlisting}
    size_t region_input_size(void);
    int region_input_read(void *dest, size_t size, off_t offset);
    int region_input_write(void *src, size_t size, off_t offset);
    void *region_input_base(void);
\end{lstlisting}
In general, generated
source and header files will have the names ``{\it name}\_driver.c'' and
``{\it name}\_driver.h'', where {\it name} is the cell's name.

  \section{\label{sec:capdlgen}CapDL Generation}
Once the user sources and the code generated by the above process have been
compiled and linked into an ELF image, \skcapdl~can be used to generate a CapDL
specification of the system.
To do this, run the tool as previously, but pass the ELF files for each cell as
command-line arguments as well as a different {\tt --output} option:
\code{
    sk-capdl --xml spec.xml --output capdl --elf cell1.bin --elf cell2.bin ...
}
This generates a single file, spec.cdl, describing the object and capability
configuration necessary to instantiate the system.

  \section{\label{sec:configgen}Scheduler Configuration Generation}
The kernel's scheduler needs a schedule describing how the various domains of
the system are to be scheduled. This is generated from the \skentity{runtime}
and \skentity{scheduleRate} attributes in the XML input. To generate a concrete
schedule, the tool also needs to be informed how many time slices occur
per-second, which is done with the {\tt --tick} option:
\code{
    sk-capdl --xml spec.xml --output config --tick 1000
}
This generates a file config.c describing the scheduling policy:
\begin{lstlisting}
#include <object/structures.h>
#include <model/statedata.h>

const dschedule_t ksDomSchedule[] = {
    { .domain = 0, .length = 280},
    { .domain = 1, .length = 420},
};

const unsigned int ksDomScheduleLength
    = sizeof(ksDomSchedule) / sizeof(dschedule_t);
\end{lstlisting}
This file is compiled into the kernel at build time to parameterise the
scheduler.

Note that it is assumed that each cell maps to a single domain. It is also
important to note that the initialiser that boots the system shares the first
domain in the schedule with the first cell. The initialiser suspends itself
after completion, so should not interfere with the remaining schedule once it
has finished booting the system.

  \section{\label{sec:extraopts}Other Options}
The \skcapdl~tool has some further command-line arguments for selecting a particular target architecture and for debugging.
To see usage information for these run:
\code{
    sk-capdl --help
}

  \chapter{\label{s:init}Initialising the System}
To initialise a system constructed as described, the ELF files for the cells need to be linked into a final image with the CapDL initialiser, a root task designed to instantiate a CapDL specification as a running system on seL4.
For details on how to perform this process refer to the CapDL initialiser documentation.

  \cleardoublepage
  \bibliographystyle{alpha}
  \bibliography{references}

\end{document}
